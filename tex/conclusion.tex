Molecular dynamics is a numerical method that can be used to study many-particle systems of molecules, clusters and macroscopic systems as solids, liquids and gases. In this project we simulated a system of argon atoms by implementing a program with many classes and a complicated class structure. Some theory around diffusion and temperature calculations from kinetic energy via the equipartition theorem were presented. The class structure was explained and the advantages of periodic boundary conditions were discussed. We used Ovito and saw how the system was a solid at certain temperatures and melted to a liquid at higher temperatures. We used diffusion through displacement of the atoms and visualized the system in Ovito to find the melting temperature. 

We found the melting temperature to be around 270-280 K but some low diffusion constants above these temperatures might indicate some strange behaviors in our model or just consequences of randoms seeds in the initial velocity. We also found the diffusion constant to be on the order of $\sim 10^{-9}$ cm$^2$/s for the solid phase and $\sim 10^{-5}$ cm$^2$/s for the liquid phase. The sudden change in diffusivity was how we found a temperature interval for the melting point. We also saw how the total energy stayed conserved while exchanging between potential energy and kinetic energy in the start, implying that the Velocity Verlet method was conserving the energy. 

Further on we could also calculate the pressure and maybe change it, to see how the melting point is dependent on the pressure. Then we could check the melting point result in this project with experimental values. By changing the density, we could probably see the phase transition to gas from liquid as well. 
