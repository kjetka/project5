Molecular dynamics is a numerical method that can be used to study many-particle systems of molecules, clusters and macroscopic systems as solids, liquids and gases, such as in this project. It is used in different science fields, physics, chemistry and biology. In this project we simulated a system of argon atoms by implementing a program with many classes and a complicated class structure. The movement was calculated with the Velocity Verlet method. 

We saw how the system was a solid at certain temperatures and melted to a liquid at higher temperatures. We used diffusion through displacement of the atoms and visualized the system in Ovito to find the melting temperature. We found the melting temperature to be around 270-280 K but some low diffusion constants above these temperatures might indicate some strange behaviors in our model. We also found the diffusion constant to be on the order of $\sim 10^{-9}$ cm$^2$/s for the solid phase and $\sim 10^{-5}$ cm$^2$/s for the liquid phase, and saw how the total energy stayed conserved while exchanging between potential energy and kinetic energy in the start. 